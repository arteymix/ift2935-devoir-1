\documentclass{article}

\usepackage[utf8]{inputenc}
\usepackage[T1]{fontenc}
\usepackage[french]{babel}

\usepackage{listings}

\title{IFT2015 Base de données \\ Devoir 1}
\author{Vincent Antaki \& Guillaume Poirier-Morency}

\begin{document}

  \maketitle

  \abstract
  Le devoir consiste à modéliser par diagramme Modèle-Entité-Relation deux cas
  pratiques.

  Afin d'exprimer explicitement nos contraintes d'intégrité, nous utilisons le
  langage OCL.

  \section{Numéro 1}

  \subsection{}
  Un individu ne peut pas être marié à lui-même.
  \begin{lstlisting}[language=OCL]
  context Mariage
      inv: self.epoux <> self.epouse
  \end{lstlisting}

  \subsection{}
  L'époux dans un mariage doit être un homme et l'épouse, une femme.
  \begin{lstlisting}[language=OCL]
  context Mariage
      inv: self.epoux.sexe = "homme" and self.epouse.sexe = "femme"
  \end{lstlisting}

  \section{}

  \subsection{}
  Un artiste ne peut pas publier un oeuvre pour un courant auquel il ne
  participe pas, mais il peut être associé à un courant même si aucune de ses
  oeuvre n'y figure.
  \begin{lstlisting}[language=OCL]
  context Artiste
      self.oeuvres->forAll(o : Oeuvre | self.courants->contains(o.courant))
  \end{lstlisting}

  Techniquement, les artistes doivent être préalablement associé à un courant
  avant d'enregistrer une certaine de ses oeuvres pour ce courant.

  \subsection{}
\end{document}
