\documentclass{article}

\usepackage[utf8]{inputenc}
\usepackage[T1]{fontenc}
\usepackage[french]{babel}

\usepackage{adjustbox}
\usepackage{fullpage}
\usepackage{rotating}
\usepackage{tikz}

\usepackage{listings}

\title{IFT2935 Base de données \\ Devoir 1}
\author{Vincent Antaki p1038646 \& Guillaume Poirier-Morency p1053380}

\begin{document}

  \maketitle

  \abstract
  Le devoir consiste à modéliser par diagramme Modèle-Entité-Relation deux cas
  pratiques.

  Afin d'exprimer explicitement nos contraintes d'intégrité, nous utilisons le
  langage OCL.

  Les attributs qui peuvent prendre des valeurs multiples (ex. les prénoms d'un
  individu pour l'arbre généalogique) sont notés par un double cercle.

  \section{Arbre généalogique}

  \begin{sidewaysfigure}
  \begin{adjustbox}{max size={\textwidth}{\textheight}}
  % Graphic for TeX using PGF
% Title: /home/guillaume/Documents/Université de Montréal/Hiver 2015/Base de donnée/devoir-1/diagramme-numero-1.dia
% Creator: Dia v0.97.3
% CreationDate: Mon Jan 19 19:20:08 2015
% For: guillaume
% \usepackage{tikz}
% The following commands are not supported in PSTricks at present
% We define them conditionally, so when they are implemented,
% this pgf file will use them.
\ifx\du\undefined
  \newlength{\du}
\fi
\setlength{\du}{15\unitlength}
\begin{tikzpicture}
\pgftransformxscale{1.000000}
\pgftransformyscale{-1.000000}
\definecolor{dialinecolor}{rgb}{0.000000, 0.000000, 0.000000}
\pgfsetstrokecolor{dialinecolor}
\definecolor{dialinecolor}{rgb}{1.000000, 1.000000, 1.000000}
\pgfsetfillcolor{dialinecolor}
\definecolor{dialinecolor}{rgb}{1.000000, 1.000000, 1.000000}
\pgfsetfillcolor{dialinecolor}
\fill (17.150000\du,10.100000\du)--(17.150000\du,11.900000\du)--(21.630000\du,11.900000\du)--(21.630000\du,10.100000\du)--cycle;
\pgfsetlinewidth{0.100000\du}
\pgfsetdash{}{0pt}
\pgfsetmiterjoin
\definecolor{dialinecolor}{rgb}{0.000000, 0.000000, 0.000000}
\pgfsetstrokecolor{dialinecolor}
\draw (17.150000\du,10.100000\du)--(17.150000\du,11.900000\du)--(21.630000\du,11.900000\du)--(21.630000\du,10.100000\du)--cycle;
% setfont left to latex
\definecolor{dialinecolor}{rgb}{0.000000, 0.000000, 0.000000}
\pgfsetstrokecolor{dialinecolor}
\node at (19.390000\du,11.200000\du){Individu};
\definecolor{dialinecolor}{rgb}{1.000000, 1.000000, 1.000000}
\pgfsetfillcolor{dialinecolor}
\fill (41.850000\du,11.450000\du)--(41.850000\du,13.250000\du)--(45.945000\du,13.250000\du)--(45.945000\du,11.450000\du)--cycle;
\pgfsetlinewidth{0.100000\du}
\pgfsetdash{}{0pt}
\pgfsetmiterjoin
\definecolor{dialinecolor}{rgb}{0.000000, 0.000000, 0.000000}
\pgfsetstrokecolor{dialinecolor}
\draw (41.850000\du,11.450000\du)--(41.850000\du,13.250000\du)--(45.945000\du,13.250000\du)--(45.945000\du,11.450000\du)--cycle;
% setfont left to latex
\definecolor{dialinecolor}{rgb}{0.000000, 0.000000, 0.000000}
\pgfsetstrokecolor{dialinecolor}
\node at (43.897500\du,12.550000\du){Mariage};
\definecolor{dialinecolor}{rgb}{1.000000, 1.000000, 1.000000}
\pgfsetfillcolor{dialinecolor}
\fill (31.750000\du,10.374000\du)--(34.290000\du,8.850000\du)--(36.830000\du,10.374000\du)--(34.290000\du,11.898000\du)--cycle;
\pgfsetlinewidth{0.100000\du}
\pgfsetdash{}{0pt}
\pgfsetmiterjoin
\definecolor{dialinecolor}{rgb}{0.000000, 0.000000, 0.000000}
\pgfsetstrokecolor{dialinecolor}
\draw (31.750000\du,10.374000\du)--(34.290000\du,8.850000\du)--(36.830000\du,10.374000\du)--(34.290000\du,11.898000\du)--cycle;
% setfont left to latex
\definecolor{dialinecolor}{rgb}{0.000000, 0.000000, 0.000000}
\pgfsetstrokecolor{dialinecolor}
\node[anchor=east] at (31.450000\du,10.074000\du){0:n};
\definecolor{dialinecolor}{rgb}{0.000000, 0.000000, 0.000000}
\pgfsetstrokecolor{dialinecolor}
\node[anchor=west] at (37.130000\du,10.074000\du){2:2};
\definecolor{dialinecolor}{rgb}{0.000000, 0.000000, 0.000000}
\pgfsetstrokecolor{dialinecolor}
\node at (34.290000\du,10.574000\du){s'engage};
\pgfsetlinewidth{0.100000\du}
\pgfsetdash{}{0pt}
\pgfsetmiterjoin
\pgfsetbuttcap
\definecolor{dialinecolor}{rgb}{0.000000, 0.000000, 0.000000}
\pgfsetstrokecolor{dialinecolor}
\draw (21.680280\du,11.000000\du)--(26.690216\du,11.000000\du)--(26.690216\du,10.374000\du)--(31.700153\du,10.374000\du);
\pgfsetlinewidth{0.100000\du}
\pgfsetdash{}{0pt}
\pgfsetmiterjoin
\pgfsetbuttcap
\definecolor{dialinecolor}{rgb}{0.000000, 0.000000, 0.000000}
\pgfsetstrokecolor{dialinecolor}
\draw (36.879847\du,10.374000\du)--(39.339795\du,10.374000\du)--(39.339795\du,12.350000\du)--(41.799744\du,12.350000\du);
\definecolor{dialinecolor}{rgb}{1.000000, 1.000000, 1.000000}
\pgfsetfillcolor{dialinecolor}
\pgfpathellipse{\pgfpoint{39.178600\du}{2.880840\du}}{\pgfpoint{3.887500\du}{0\du}}{\pgfpoint{0\du}{0.900000\du}}
\pgfusepath{fill}
\pgfsetlinewidth{0.100000\du}
\pgfsetdash{}{0pt}
\definecolor{dialinecolor}{rgb}{0.000000, 0.000000, 0.000000}
\pgfsetstrokecolor{dialinecolor}
\pgfpathellipse{\pgfpoint{39.178600\du}{2.880840\du}}{\pgfpoint{3.887500\du}{0\du}}{\pgfpoint{0\du}{0.900000\du}}
\pgfusepath{stroke}
% setfont left to latex
\definecolor{dialinecolor}{rgb}{0.000000, 0.000000, 0.000000}
\pgfsetstrokecolor{dialinecolor}
\node at (39.178600\du,3.080840\du){date du mariage};
\definecolor{dialinecolor}{rgb}{1.000000, 1.000000, 1.000000}
\pgfsetfillcolor{dialinecolor}
\pgfpathellipse{\pgfpoint{23.427500\du}{1.200000\du}}{\pgfpoint{1.577500\du}{0\du}}{\pgfpoint{0\du}{0.900000\du}}
\pgfusepath{fill}
\pgfsetlinewidth{0.100000\du}
\pgfsetdash{}{0pt}
\definecolor{dialinecolor}{rgb}{0.000000, 0.000000, 0.000000}
\pgfsetstrokecolor{dialinecolor}
\pgfpathellipse{\pgfpoint{23.427500\du}{1.200000\du}}{\pgfpoint{1.577500\du}{0\du}}{\pgfpoint{0\du}{0.900000\du}}
\pgfusepath{stroke}
% setfont left to latex
\definecolor{dialinecolor}{rgb}{0.000000, 0.000000, 0.000000}
\pgfsetstrokecolor{dialinecolor}
\node at (23.427500\du,1.400000\du){nom};
\definecolor{dialinecolor}{rgb}{1.000000, 1.000000, 1.000000}
\pgfsetfillcolor{dialinecolor}
\pgfpathellipse{\pgfpoint{30.647500\du}{3.250000\du}}{\pgfpoint{2.347500\du}{0\du}}{\pgfpoint{0\du}{0.900000\du}}
\pgfusepath{fill}
\pgfsetlinewidth{0.100000\du}
\pgfsetdash{}{0pt}
\definecolor{dialinecolor}{rgb}{0.000000, 0.000000, 0.000000}
\pgfsetstrokecolor{dialinecolor}
\pgfpathellipse{\pgfpoint{30.647500\du}{3.250000\du}}{\pgfpoint{2.347500\du}{0\du}}{\pgfpoint{0\du}{0.900000\du}}
\pgfusepath{stroke}
\definecolor{dialinecolor}{rgb}{0.000000, 0.000000, 0.000000}
\pgfsetstrokecolor{dialinecolor}
\pgfpathellipse{\pgfpoint{30.647500\du}{3.250000\du}}{\pgfpoint{1.947500\du}{0\du}}{\pgfpoint{0\du}{0.700000\du}}
\pgfusepath{stroke}
% setfont left to latex
\definecolor{dialinecolor}{rgb}{0.000000, 0.000000, 0.000000}
\pgfsetstrokecolor{dialinecolor}
\node at (30.647500\du,3.450000\du){prénoms};
\definecolor{dialinecolor}{rgb}{1.000000, 1.000000, 1.000000}
\pgfsetfillcolor{dialinecolor}
\pgfpathellipse{\pgfpoint{28.820000\du}{16.500000\du}}{\pgfpoint{1.770000\du}{0\du}}{\pgfpoint{0\du}{0.900000\du}}
\pgfusepath{fill}
\pgfsetlinewidth{0.100000\du}
\pgfsetdash{}{0pt}
\definecolor{dialinecolor}{rgb}{0.000000, 0.000000, 0.000000}
\pgfsetstrokecolor{dialinecolor}
\pgfpathellipse{\pgfpoint{28.820000\du}{16.500000\du}}{\pgfpoint{1.770000\du}{0\du}}{\pgfpoint{0\du}{0.900000\du}}
\pgfusepath{stroke}
% setfont left to latex
\definecolor{dialinecolor}{rgb}{0.000000, 0.000000, 0.000000}
\pgfsetstrokecolor{dialinecolor}
\node at (28.820000\du,16.700000\du){sexe};
\definecolor{dialinecolor}{rgb}{1.000000, 1.000000, 1.000000}
\pgfsetfillcolor{dialinecolor}
\pgfpathellipse{\pgfpoint{22.272500\du}{18.700000\du}}{\pgfpoint{4.272500\du}{0\du}}{\pgfpoint{0\du}{0.900000\du}}
\pgfusepath{fill}
\pgfsetlinewidth{0.100000\du}
\pgfsetdash{}{0pt}
\definecolor{dialinecolor}{rgb}{0.000000, 0.000000, 0.000000}
\pgfsetstrokecolor{dialinecolor}
\pgfpathellipse{\pgfpoint{22.272500\du}{18.700000\du}}{\pgfpoint{4.272500\du}{0\du}}{\pgfpoint{0\du}{0.900000\du}}
\pgfusepath{stroke}
% setfont left to latex
\definecolor{dialinecolor}{rgb}{0.000000, 0.000000, 0.000000}
\pgfsetstrokecolor{dialinecolor}
\node at (22.272500\du,18.900000\du){date de naissance};
\definecolor{dialinecolor}{rgb}{1.000000, 1.000000, 1.000000}
\pgfsetfillcolor{dialinecolor}
\pgfpathellipse{\pgfpoint{12.522500\du}{18.650000\du}}{\pgfpoint{4.272500\du}{0\du}}{\pgfpoint{0\du}{0.900000\du}}
\pgfusepath{fill}
\pgfsetlinewidth{0.100000\du}
\pgfsetdash{}{0pt}
\definecolor{dialinecolor}{rgb}{0.000000, 0.000000, 0.000000}
\pgfsetstrokecolor{dialinecolor}
\pgfpathellipse{\pgfpoint{12.522500\du}{18.650000\du}}{\pgfpoint{4.272500\du}{0\du}}{\pgfpoint{0\du}{0.900000\du}}
\pgfusepath{stroke}
% setfont left to latex
\definecolor{dialinecolor}{rgb}{0.000000, 0.000000, 0.000000}
\pgfsetstrokecolor{dialinecolor}
\node at (12.522500\du,18.850000\du){lieu de naissance};
\definecolor{dialinecolor}{rgb}{1.000000, 1.000000, 1.000000}
\pgfsetfillcolor{dialinecolor}
\pgfpathellipse{\pgfpoint{38.053900\du}{18.733600\du}}{\pgfpoint{3.887500\du}{0\du}}{\pgfpoint{0\du}{0.900000\du}}
\pgfusepath{fill}
\pgfsetlinewidth{0.100000\du}
\pgfsetdash{}{0pt}
\definecolor{dialinecolor}{rgb}{0.000000, 0.000000, 0.000000}
\pgfsetstrokecolor{dialinecolor}
\pgfpathellipse{\pgfpoint{38.053900\du}{18.733600\du}}{\pgfpoint{3.887500\du}{0\du}}{\pgfpoint{0\du}{0.900000\du}}
\pgfusepath{stroke}
% setfont left to latex
\definecolor{dialinecolor}{rgb}{0.000000, 0.000000, 0.000000}
\pgfsetstrokecolor{dialinecolor}
\node at (38.053900\du,18.933600\du){date de divorce};
\pgfsetlinewidth{0.100000\du}
\pgfsetdash{}{0pt}
\pgfsetmiterjoin
\pgfsetbuttcap
\definecolor{dialinecolor}{rgb}{0.000000, 0.000000, 0.000000}
\pgfsetstrokecolor{dialinecolor}
\draw (19.390000\du,10.049536\du)--(19.390000\du,7.125000\du)--(30.647500\du,7.125000\du)--(30.647500\du,4.200464\du);
\pgfsetlinewidth{0.100000\du}
\pgfsetdash{}{0pt}
\pgfsetmiterjoin
\pgfsetbuttcap
\definecolor{dialinecolor}{rgb}{0.000000, 0.000000, 0.000000}
\pgfsetstrokecolor{dialinecolor}
\draw (19.390000\du,10.049536\du)--(19.390000\du,6.100000\du)--(23.427500\du,6.100000\du)--(23.427500\du,2.150464\du);
\pgfsetlinewidth{0.100000\du}
\pgfsetdash{}{0pt}
\pgfsetmiterjoin
\pgfsetbuttcap
\definecolor{dialinecolor}{rgb}{0.000000, 0.000000, 0.000000}
\pgfsetstrokecolor{dialinecolor}
\draw (12.522500\du,17.699536\du)--(12.522500\du,14.825000\du)--(19.390000\du,14.825000\du)--(19.390000\du,11.950464\du);
\pgfsetlinewidth{0.100000\du}
\pgfsetdash{}{0pt}
\pgfsetmiterjoin
\pgfsetbuttcap
\definecolor{dialinecolor}{rgb}{0.000000, 0.000000, 0.000000}
\pgfsetstrokecolor{dialinecolor}
\draw (19.390000\du,11.950464\du)--(19.390000\du,14.850000\du)--(22.272500\du,14.850000\du)--(22.272500\du,17.749536\du);
\pgfsetlinewidth{0.100000\du}
\pgfsetdash{}{0pt}
\pgfsetmiterjoin
\pgfsetbuttcap
\definecolor{dialinecolor}{rgb}{0.000000, 0.000000, 0.000000}
\pgfsetstrokecolor{dialinecolor}
\draw (19.390000\du,11.950464\du)--(19.390000\du,13.750000\du)--(28.820000\du,13.750000\du)--(28.820000\du,15.549536\du);
\pgfsetlinewidth{0.100000\du}
\pgfsetdash{}{0pt}
\pgfsetmiterjoin
\pgfsetbuttcap
\definecolor{dialinecolor}{rgb}{0.000000, 0.000000, 0.000000}
\pgfsetstrokecolor{dialinecolor}
\draw (43.897500\du,13.300195\du)--(43.897500\du,15.516800\du)--(38.053900\du,15.516800\du)--(38.053900\du,17.784110\du);
\pgfsetlinewidth{0.100000\du}
\pgfsetdash{}{0pt}
\pgfsetmiterjoin
\pgfsetbuttcap
\definecolor{dialinecolor}{rgb}{0.000000, 0.000000, 0.000000}
\pgfsetstrokecolor{dialinecolor}
\draw (39.178600\du,3.831304\du)--(39.178600\du,7.615420\du)--(43.897500\du,7.615420\du)--(43.897500\du,11.399536\du);
\definecolor{dialinecolor}{rgb}{1.000000, 1.000000, 1.000000}
\pgfsetfillcolor{dialinecolor}
\fill (1.820560\du,5.369650\du)--(5.130560\du,3.383650\du)--(8.440560\du,5.369650\du)--(5.130560\du,7.355650\du)--cycle;
\pgfsetlinewidth{0.100000\du}
\pgfsetdash{}{0pt}
\pgfsetmiterjoin
\definecolor{dialinecolor}{rgb}{0.000000, 0.000000, 0.000000}
\pgfsetstrokecolor{dialinecolor}
\draw (1.820560\du,5.369650\du)--(5.130560\du,3.383650\du)--(8.440560\du,5.369650\du)--(5.130560\du,7.355650\du)--cycle;
% setfont left to latex
\definecolor{dialinecolor}{rgb}{0.000000, 0.000000, 0.000000}
\pgfsetstrokecolor{dialinecolor}
\node[anchor=east] at (1.520560\du,5.069650\du){0:n};
\definecolor{dialinecolor}{rgb}{0.000000, 0.000000, 0.000000}
\pgfsetstrokecolor{dialinecolor}
\node[anchor=west] at (8.740560\du,5.069650\du){0:2};
\definecolor{dialinecolor}{rgb}{0.000000, 0.000000, 0.000000}
\pgfsetstrokecolor{dialinecolor}
\node at (5.130560\du,5.569650\du){sont parents};
\pgfsetlinewidth{0.100000\du}
\pgfsetdash{}{0pt}
\pgfsetmiterjoin
\pgfsetbuttcap
\definecolor{dialinecolor}{rgb}{0.000000, 0.000000, 0.000000}
\pgfsetstrokecolor{dialinecolor}
\draw (8.487305\du,5.369650\du)--(11.837500\du,5.369650\du)--(11.837500\du,11.000000\du)--(17.101755\du,11.000000\du);
\pgfsetlinewidth{0.100000\du}
\pgfsetdash{}{0pt}
\pgfsetmiterjoin
\pgfsetbuttcap
\definecolor{dialinecolor}{rgb}{0.000000, 0.000000, 0.000000}
\pgfsetstrokecolor{dialinecolor}
\draw (1.771229\du,5.369650\du)--(-1.297760\du,5.369650\du)--(-1.297760\du,11.000000\du)--(17.100760\du,11.000000\du);
\end{tikzpicture}

  \end{adjustbox}
  \caption{Diagramme Modèle-entité-relation pour l'arbre généalogique.}
  \end{sidewaysfigure}

  La relation « est parent » définit la relation « est l'enfant » tout en
  imposant la cardinalité de deux parents biologiques par enfant. Spécifier
  uniquement la deuxième relation aurait nécéssité une contrainte d'intégrité.

  \subsection{}
  Un individu ne peut pas être marié à lui-même.
  \begin{lstlisting}[language=OCL]
  context Mariage
      inv: self.epoux <> self.epouse
  \end{lstlisting}

  \subsection{}
  L'époux dans un mariage doit être un homme et l'épouse, une femme.
  \begin{lstlisting}[language=OCL]
  context Mariage
      inv: self.epoux.sexe = "homme" and self.epouse.sexe = "femme"
  \end{lstlisting}

  \section{Inventaire des oeuvres d'arts}

  \begin{sidewaysfigure}
  \begin{adjustbox}{max size={\textwidth}{\textheight}}
  % Graphic for TeX using PGF
% Title: /home/guillaume/Documents/Université de Montréal/Hiver 2015/Base de donnée/devoir-1/diagramme-numero-2.dia
% Creator: Dia v0.97.3
% CreationDate: Mon Jan 19 19:20:16 2015
% For: guillaume
% \usepackage{tikz}
% The following commands are not supported in PSTricks at present
% We define them conditionally, so when they are implemented,
% this pgf file will use them.
\ifx\du\undefined
  \newlength{\du}
\fi
\setlength{\du}{15\unitlength}
\begin{tikzpicture}
\pgftransformxscale{1.000000}
\pgftransformyscale{-1.000000}
\definecolor{dialinecolor}{rgb}{0.000000, 0.000000, 0.000000}
\pgfsetstrokecolor{dialinecolor}
\definecolor{dialinecolor}{rgb}{1.000000, 1.000000, 1.000000}
\pgfsetfillcolor{dialinecolor}
\definecolor{dialinecolor}{rgb}{1.000000, 1.000000, 1.000000}
\pgfsetfillcolor{dialinecolor}
\fill (11.160700\du,4.081080\du)--(11.160700\du,5.881080\du)--(14.870700\du,5.881080\du)--(14.870700\du,4.081080\du)--cycle;
\pgfsetlinewidth{0.100000\du}
\pgfsetdash{}{0pt}
\pgfsetmiterjoin
\definecolor{dialinecolor}{rgb}{0.000000, 0.000000, 0.000000}
\pgfsetstrokecolor{dialinecolor}
\draw (11.160700\du,4.081080\du)--(11.160700\du,5.881080\du)--(14.870700\du,5.881080\du)--(14.870700\du,4.081080\du)--cycle;
% setfont left to latex
\definecolor{dialinecolor}{rgb}{0.000000, 0.000000, 0.000000}
\pgfsetstrokecolor{dialinecolor}
\node at (13.015700\du,5.181080\du){Oeuvre};
\definecolor{dialinecolor}{rgb}{1.000000, 1.000000, 1.000000}
\pgfsetfillcolor{dialinecolor}
\fill (34.787600\du,-2.413520\du)--(34.787600\du,-0.613520\du)--(43.117600\du,-0.613520\du)--(43.117600\du,-2.413520\du)--cycle;
\pgfsetlinewidth{0.100000\du}
\pgfsetdash{}{0pt}
\pgfsetmiterjoin
\definecolor{dialinecolor}{rgb}{0.000000, 0.000000, 0.000000}
\pgfsetstrokecolor{dialinecolor}
\draw (34.787600\du,-2.413520\du)--(34.787600\du,-0.613520\du)--(43.117600\du,-0.613520\du)--(43.117600\du,-2.413520\du)--cycle;
% setfont left to latex
\definecolor{dialinecolor}{rgb}{0.000000, 0.000000, 0.000000}
\pgfsetstrokecolor{dialinecolor}
\node at (38.952600\du,-1.313520\du){Courant artistique};
\definecolor{dialinecolor}{rgb}{1.000000, 1.000000, 1.000000}
\pgfsetfillcolor{dialinecolor}
\fill (33.183400\du,4.370290\du)--(33.183400\du,6.170290\du)--(37.278400\du,6.170290\du)--(37.278400\du,4.370290\du)--cycle;
\pgfsetlinewidth{0.100000\du}
\pgfsetdash{}{0pt}
\pgfsetmiterjoin
\definecolor{dialinecolor}{rgb}{0.000000, 0.000000, 0.000000}
\pgfsetstrokecolor{dialinecolor}
\draw (33.183400\du,4.370290\du)--(33.183400\du,6.170290\du)--(37.278400\du,6.170290\du)--(37.278400\du,4.370290\du)--cycle;
% setfont left to latex
\definecolor{dialinecolor}{rgb}{0.000000, 0.000000, 0.000000}
\pgfsetstrokecolor{dialinecolor}
\node at (35.230900\du,5.470290\du){Artiste};
\definecolor{dialinecolor}{rgb}{1.000000, 1.000000, 1.000000}
\pgfsetfillcolor{dialinecolor}
\pgfpathellipse{\pgfpoint{29.876200\du}{10.818750\du}}{\pgfpoint{1.577500\du}{0\du}}{\pgfpoint{0\du}{0.900000\du}}
\pgfusepath{fill}
\pgfsetlinewidth{0.100000\du}
\pgfsetdash{}{0pt}
\definecolor{dialinecolor}{rgb}{0.000000, 0.000000, 0.000000}
\pgfsetstrokecolor{dialinecolor}
\pgfpathellipse{\pgfpoint{29.876200\du}{10.818750\du}}{\pgfpoint{1.577500\du}{0\du}}{\pgfpoint{0\du}{0.900000\du}}
\pgfusepath{stroke}
% setfont left to latex
\definecolor{dialinecolor}{rgb}{0.000000, 0.000000, 0.000000}
\pgfsetstrokecolor{dialinecolor}
\node at (29.876200\du,11.018750\du){nom};
\definecolor{dialinecolor}{rgb}{1.000000, 1.000000, 1.000000}
\pgfsetfillcolor{dialinecolor}
\pgfpathellipse{\pgfpoint{35.659100\du}{11.422800\du}}{\pgfpoint{2.155000\du}{0\du}}{\pgfpoint{0\du}{0.900000\du}}
\pgfusepath{fill}
\pgfsetlinewidth{0.100000\du}
\pgfsetdash{}{0pt}
\definecolor{dialinecolor}{rgb}{0.000000, 0.000000, 0.000000}
\pgfsetstrokecolor{dialinecolor}
\pgfpathellipse{\pgfpoint{35.659100\du}{11.422800\du}}{\pgfpoint{2.155000\du}{0\du}}{\pgfpoint{0\du}{0.900000\du}}
\pgfusepath{stroke}
% setfont left to latex
\definecolor{dialinecolor}{rgb}{0.000000, 0.000000, 0.000000}
\pgfsetstrokecolor{dialinecolor}
\node at (35.659100\du,11.622800\du){prénom};
\definecolor{dialinecolor}{rgb}{1.000000, 1.000000, 1.000000}
\pgfsetfillcolor{dialinecolor}
\pgfpathellipse{\pgfpoint{43.889100\du}{12.116100\du}}{\pgfpoint{3.117500\du}{0\du}}{\pgfpoint{0\du}{0.900000\du}}
\pgfusepath{fill}
\pgfsetlinewidth{0.100000\du}
\pgfsetdash{}{0pt}
\definecolor{dialinecolor}{rgb}{0.000000, 0.000000, 0.000000}
\pgfsetstrokecolor{dialinecolor}
\pgfpathellipse{\pgfpoint{43.889100\du}{12.116100\du}}{\pgfpoint{3.117500\du}{0\du}}{\pgfpoint{0\du}{0.900000\du}}
\pgfusepath{stroke}
% setfont left to latex
\definecolor{dialinecolor}{rgb}{0.000000, 0.000000, 0.000000}
\pgfsetstrokecolor{dialinecolor}
\node at (43.889100\du,12.316100\du){nationalité};
\definecolor{dialinecolor}{rgb}{1.000000, 1.000000, 1.000000}
\pgfsetfillcolor{dialinecolor}
\pgfpathellipse{\pgfpoint{47.571200\du}{8.220150\du}}{\pgfpoint{4.272500\du}{0\du}}{\pgfpoint{0\du}{0.900000\du}}
\pgfusepath{fill}
\pgfsetlinewidth{0.100000\du}
\pgfsetdash{}{0pt}
\definecolor{dialinecolor}{rgb}{0.000000, 0.000000, 0.000000}
\pgfsetstrokecolor{dialinecolor}
\pgfpathellipse{\pgfpoint{47.571200\du}{8.220150\du}}{\pgfpoint{4.272500\du}{0\du}}{\pgfpoint{0\du}{0.900000\du}}
\pgfusepath{stroke}
% setfont left to latex
\definecolor{dialinecolor}{rgb}{0.000000, 0.000000, 0.000000}
\pgfsetstrokecolor{dialinecolor}
\node at (47.571200\du,8.420150\du){date de naissance};
\definecolor{dialinecolor}{rgb}{1.000000, 1.000000, 1.000000}
\pgfsetfillcolor{dialinecolor}
\pgfpathellipse{\pgfpoint{49.132000\du}{3.413360\du}}{\pgfpoint{3.502500\du}{0\du}}{\pgfpoint{0\du}{0.900000\du}}
\pgfusepath{fill}
\pgfsetlinewidth{0.100000\du}
\pgfsetdash{}{0pt}
\definecolor{dialinecolor}{rgb}{0.000000, 0.000000, 0.000000}
\pgfsetstrokecolor{dialinecolor}
\pgfpathellipse{\pgfpoint{49.132000\du}{3.413360\du}}{\pgfpoint{3.502500\du}{0\du}}{\pgfpoint{0\du}{0.900000\du}}
\pgfusepath{stroke}
% setfont left to latex
\definecolor{dialinecolor}{rgb}{0.000000, 0.000000, 0.000000}
\pgfsetstrokecolor{dialinecolor}
\node at (49.132000\du,3.613360\du){date de décès};
\pgfsetlinewidth{0.100000\du}
\pgfsetdash{}{0pt}
\pgfsetmiterjoin
\pgfsetbuttcap
\definecolor{dialinecolor}{rgb}{0.000000, 0.000000, 0.000000}
\pgfsetstrokecolor{dialinecolor}
\draw (35.230900\du,6.220754\du)--(35.230900\du,8.044520\du)--(29.876200\du,8.044520\du)--(29.876200\du,9.868286\du);
\pgfsetlinewidth{0.100000\du}
\pgfsetdash{}{0pt}
\pgfsetmiterjoin
\pgfsetbuttcap
\definecolor{dialinecolor}{rgb}{0.000000, 0.000000, 0.000000}
\pgfsetstrokecolor{dialinecolor}
\draw (37.328656\du,5.270290\du)--(39.024935\du,5.270290\du)--(39.024935\du,12.116100\du)--(40.721213\du,12.116100\du);
\pgfsetlinewidth{0.100000\du}
\pgfsetdash{}{0pt}
\pgfsetmiterjoin
\pgfsetbuttcap
\definecolor{dialinecolor}{rgb}{0.000000, 0.000000, 0.000000}
\pgfsetstrokecolor{dialinecolor}
\draw (37.328656\du,5.270290\du)--(40.288546\du,5.270290\du)--(40.288546\du,8.220150\du)--(43.248436\du,8.220150\du);
\pgfsetlinewidth{0.100000\du}
\pgfsetdash{}{0pt}
\pgfsetmiterjoin
\pgfsetbuttcap
\definecolor{dialinecolor}{rgb}{0.000000, 0.000000, 0.000000}
\pgfsetstrokecolor{dialinecolor}
\draw (37.328095\du,5.270290\du)--(40.558100\du,5.270290\du)--(40.558100\du,3.413360\du)--(45.579253\du,3.413360\du);
\pgfsetlinewidth{0.100000\du}
\pgfsetdash{}{0pt}
\pgfsetmiterjoin
\pgfsetbuttcap
\definecolor{dialinecolor}{rgb}{0.000000, 0.000000, 0.000000}
\pgfsetstrokecolor{dialinecolor}
\draw (35.659100\du,10.472336\du)--(35.659100\du,8.346545\du)--(35.230900\du,8.346545\du)--(35.230900\du,6.220754\du);
\definecolor{dialinecolor}{rgb}{1.000000, 1.000000, 1.000000}
\pgfsetfillcolor{dialinecolor}
\fill (21.670300\du,7.176630\du)--(24.210300\du,5.652630\du)--(26.750300\du,7.176630\du)--(24.210300\du,8.700630\du)--cycle;
\pgfsetlinewidth{0.100000\du}
\pgfsetdash{}{0pt}
\pgfsetmiterjoin
\definecolor{dialinecolor}{rgb}{0.000000, 0.000000, 0.000000}
\pgfsetstrokecolor{dialinecolor}
\draw (21.670300\du,7.176630\du)--(24.210300\du,5.652630\du)--(26.750300\du,7.176630\du)--(24.210300\du,8.700630\du)--cycle;
% setfont left to latex
\definecolor{dialinecolor}{rgb}{0.000000, 0.000000, 0.000000}
\pgfsetstrokecolor{dialinecolor}
\node[anchor=east] at (21.370300\du,6.876630\du){1:n};
\definecolor{dialinecolor}{rgb}{0.000000, 0.000000, 0.000000}
\pgfsetstrokecolor{dialinecolor}
\node[anchor=west] at (27.050300\du,6.876630\du){1:1};
\definecolor{dialinecolor}{rgb}{0.000000, 0.000000, 0.000000}
\pgfsetstrokecolor{dialinecolor}
\node at (24.210300\du,7.376630\du){réalisée};
\pgfsetlinewidth{0.100000\du}
\pgfsetdash{}{0pt}
\pgfsetmiterjoin
\pgfsetbuttcap
\definecolor{dialinecolor}{rgb}{0.000000, 0.000000, 0.000000}
\pgfsetstrokecolor{dialinecolor}
\draw (21.620453\du,7.176630\du)--(18.270809\du,7.176630\du)--(18.270809\du,4.981080\du)--(14.921165\du,4.981080\du);
\pgfsetlinewidth{0.100000\du}
\pgfsetdash{}{0pt}
\pgfsetmiterjoin
\pgfsetbuttcap
\definecolor{dialinecolor}{rgb}{0.000000, 0.000000, 0.000000}
\pgfsetstrokecolor{dialinecolor}
\draw (26.800147\du,7.176630\du)--(29.966645\du,7.176630\du)--(29.966645\du,5.270290\du)--(33.133144\du,5.270290\du);
\definecolor{dialinecolor}{rgb}{1.000000, 1.000000, 1.000000}
\pgfsetfillcolor{dialinecolor}
\fill (20.153100\du,0.992180\du)--(22.885600\du,-0.647320\du)--(25.618100\du,0.992180\du)--(22.885600\du,2.631680\du)--cycle;
\pgfsetlinewidth{0.100000\du}
\pgfsetdash{}{0pt}
\pgfsetmiterjoin
\definecolor{dialinecolor}{rgb}{0.000000, 0.000000, 0.000000}
\pgfsetstrokecolor{dialinecolor}
\draw (20.153100\du,0.992180\du)--(22.885600\du,-0.647320\du)--(25.618100\du,0.992180\du)--(22.885600\du,2.631680\du)--cycle;
% setfont left to latex
\definecolor{dialinecolor}{rgb}{0.000000, 0.000000, 0.000000}
\pgfsetstrokecolor{dialinecolor}
\node[anchor=east] at (19.853100\du,0.692180\du){0:n};
\definecolor{dialinecolor}{rgb}{0.000000, 0.000000, 0.000000}
\pgfsetstrokecolor{dialinecolor}
\node[anchor=west] at (25.918100\du,0.692180\du){1:1};
\definecolor{dialinecolor}{rgb}{0.000000, 0.000000, 0.000000}
\pgfsetstrokecolor{dialinecolor}
\node at (22.885600\du,1.192180\du){s'inscrit};
\pgfsetlinewidth{0.100000\du}
\pgfsetdash{}{0pt}
\pgfsetmiterjoin
\pgfsetbuttcap
\definecolor{dialinecolor}{rgb}{0.000000, 0.000000, 0.000000}
\pgfsetstrokecolor{dialinecolor}
\draw (25.667494\du,0.992180\du)--(30.202418\du,0.992180\du)--(30.202418\du,-1.513520\du)--(34.737343\du,-1.513520\du);
\pgfsetlinewidth{0.100000\du}
\pgfsetdash{}{0pt}
\pgfsetmiterjoin
\pgfsetbuttcap
\definecolor{dialinecolor}{rgb}{0.000000, 0.000000, 0.000000}
\pgfsetstrokecolor{dialinecolor}
\draw (14.921165\du,4.981080\du)--(17.512436\du,4.981080\du)--(17.512436\du,0.992180\du)--(20.103706\du,0.992180\du);
\definecolor{dialinecolor}{rgb}{1.000000, 1.000000, 1.000000}
\pgfsetfillcolor{dialinecolor}
\pgfpathellipse{\pgfpoint{7.721070\du}{-2.752690\du}}{\pgfpoint{2.925000\du}{0\du}}{\pgfpoint{0\du}{0.900000\du}}
\pgfusepath{fill}
\pgfsetlinewidth{0.100000\du}
\pgfsetdash{}{0pt}
\definecolor{dialinecolor}{rgb}{0.000000, 0.000000, 0.000000}
\pgfsetstrokecolor{dialinecolor}
\pgfpathellipse{\pgfpoint{7.721070\du}{-2.752690\du}}{\pgfpoint{2.925000\du}{0\du}}{\pgfpoint{0\du}{0.900000\du}}
\pgfusepath{stroke}
% setfont left to latex
\definecolor{dialinecolor}{rgb}{0.000000, 0.000000, 0.000000}
\pgfsetstrokecolor{dialinecolor}
\node at (7.721070\du,-2.552690\du){dimensions};
\definecolor{dialinecolor}{rgb}{1.000000, 1.000000, 1.000000}
\pgfsetfillcolor{dialinecolor}
\pgfpathellipse{\pgfpoint{3.095560\du}{10.123060\du}}{\pgfpoint{2.540000\du}{0\du}}{\pgfpoint{0\du}{0.900000\du}}
\pgfusepath{fill}
\pgfsetlinewidth{0.100000\du}
\pgfsetdash{}{0pt}
\definecolor{dialinecolor}{rgb}{0.000000, 0.000000, 0.000000}
\pgfsetstrokecolor{dialinecolor}
\pgfpathellipse{\pgfpoint{3.095560\du}{10.123060\du}}{\pgfpoint{2.540000\du}{0\du}}{\pgfpoint{0\du}{0.900000\du}}
\pgfusepath{stroke}
\definecolor{dialinecolor}{rgb}{0.000000, 0.000000, 0.000000}
\pgfsetstrokecolor{dialinecolor}
\pgfpathellipse{\pgfpoint{3.095560\du}{10.123060\du}}{\pgfpoint{2.140000\du}{0\du}}{\pgfpoint{0\du}{0.700000\du}}
\pgfusepath{stroke}
% setfont left to latex
\definecolor{dialinecolor}{rgb}{0.000000, 0.000000, 0.000000}
\pgfsetstrokecolor{dialinecolor}
\node at (3.095560\du,10.323060\du){matières};
\definecolor{dialinecolor}{rgb}{1.000000, 1.000000, 1.000000}
\pgfsetfillcolor{dialinecolor}
\pgfpathellipse{\pgfpoint{7.964000\du}{12.638000\du}}{\pgfpoint{1.962500\du}{0\du}}{\pgfpoint{0\du}{0.900000\du}}
\pgfusepath{fill}
\pgfsetlinewidth{0.100000\du}
\pgfsetdash{}{0pt}
\definecolor{dialinecolor}{rgb}{0.000000, 0.000000, 0.000000}
\pgfsetstrokecolor{dialinecolor}
\pgfpathellipse{\pgfpoint{7.964000\du}{12.638000\du}}{\pgfpoint{1.962500\du}{0\du}}{\pgfpoint{0\du}{0.900000\du}}
\pgfusepath{stroke}
% setfont left to latex
\definecolor{dialinecolor}{rgb}{0.000000, 0.000000, 0.000000}
\pgfsetstrokecolor{dialinecolor}
\node at (7.964000\du,12.838000\du){année};
\definecolor{dialinecolor}{rgb}{1.000000, 1.000000, 1.000000}
\pgfsetfillcolor{dialinecolor}
\pgfpathellipse{\pgfpoint{17.238400\du}{11.471700\du}}{\pgfpoint{1.962500\du}{0\du}}{\pgfpoint{0\du}{0.900000\du}}
\pgfusepath{fill}
\pgfsetlinewidth{0.100000\du}
\pgfsetdash{}{0pt}
\definecolor{dialinecolor}{rgb}{0.000000, 0.000000, 0.000000}
\pgfsetstrokecolor{dialinecolor}
\pgfpathellipse{\pgfpoint{17.238400\du}{11.471700\du}}{\pgfpoint{1.962500\du}{0\du}}{\pgfpoint{0\du}{0.900000\du}}
\pgfusepath{stroke}
% setfont left to latex
\definecolor{dialinecolor}{rgb}{0.000000, 0.000000, 0.000000}
\pgfsetstrokecolor{dialinecolor}
\node at (17.238400\du,11.671700\du){titre};
\definecolor{dialinecolor}{rgb}{1.000000, 1.000000, 1.000000}
\pgfsetfillcolor{dialinecolor}
\pgfpathellipse{\pgfpoint{16.713600\du}{-4.451390\du}}{\pgfpoint{1.770000\du}{0\du}}{\pgfpoint{0\du}{0.900000\du}}
\pgfusepath{fill}
\pgfsetlinewidth{0.100000\du}
\pgfsetdash{}{0pt}
\definecolor{dialinecolor}{rgb}{0.000000, 0.000000, 0.000000}
\pgfsetstrokecolor{dialinecolor}
\pgfpathellipse{\pgfpoint{16.713600\du}{-4.451390\du}}{\pgfpoint{1.770000\du}{0\du}}{\pgfpoint{0\du}{0.900000\du}}
\pgfusepath{stroke}
% setfont left to latex
\definecolor{dialinecolor}{rgb}{0.000000, 0.000000, 0.000000}
\pgfsetstrokecolor{dialinecolor}
\node at (16.713600\du,-4.251390\du){type};
\definecolor{dialinecolor}{rgb}{1.000000, 1.000000, 1.000000}
\pgfsetfillcolor{dialinecolor}
\pgfpathellipse{\pgfpoint{-6.703200\du}{12.593200\du}}{\pgfpoint{4.657500\du}{0\du}}{\pgfpoint{0\du}{0.900000\du}}
\pgfusepath{fill}
\pgfsetlinewidth{0.100000\du}
\pgfsetdash{}{0pt}
\definecolor{dialinecolor}{rgb}{0.000000, 0.000000, 0.000000}
\pgfsetstrokecolor{dialinecolor}
\pgfpathellipse{\pgfpoint{-6.703200\du}{12.593200\du}}{\pgfpoint{4.657500\du}{0\du}}{\pgfpoint{0\du}{0.900000\du}}
\pgfusepath{stroke}
% setfont left to latex
\definecolor{dialinecolor}{rgb}{0.000000, 0.000000, 0.000000}
\pgfsetstrokecolor{dialinecolor}
\node at (-6.703200\du,12.793200\du){numéro d'exemplaire};
\pgfsetlinewidth{0.100000\du}
\pgfsetdash{}{0pt}
\pgfsetmiterjoin
\pgfsetbuttcap
\definecolor{dialinecolor}{rgb}{0.000000, 0.000000, 0.000000}
\pgfsetstrokecolor{dialinecolor}
\draw (7.721070\du,-1.802226\du)--(7.721070\du,1.114195\du)--(13.015700\du,1.114195\du)--(13.015700\du,4.030616\du);
\pgfsetlinewidth{0.100000\du}
\pgfsetdash{}{0pt}
\pgfsetmiterjoin
\pgfsetbuttcap
\definecolor{dialinecolor}{rgb}{0.000000, 0.000000, 0.000000}
\pgfsetstrokecolor{dialinecolor}
\draw (3.095560\du,9.172596\du)--(3.095560\du,7.552070\du)--(13.015700\du,7.552070\du)--(13.015700\du,5.931544\du);
\pgfsetlinewidth{0.100000\du}
\pgfsetdash{}{0pt}
\pgfsetmiterjoin
\pgfsetbuttcap
\definecolor{dialinecolor}{rgb}{0.000000, 0.000000, 0.000000}
\pgfsetstrokecolor{dialinecolor}
\draw (16.713600\du,-3.500926\du)--(16.713600\du,0.264845\du)--(13.015700\du,0.264845\du)--(13.015700\du,4.030616\du);
\pgfsetlinewidth{0.100000\du}
\pgfsetdash{}{0pt}
\pgfsetmiterjoin
\pgfsetbuttcap
\definecolor{dialinecolor}{rgb}{0.000000, 0.000000, 0.000000}
\pgfsetstrokecolor{dialinecolor}
\draw (-6.703200\du,11.642736\du)--(-6.703200\du,9.028346\du)--(-2.480390\du,9.028346\du)--(-2.480390\du,6.413956\du);
\pgfsetlinewidth{0.100000\du}
\pgfsetdash{}{0pt}
\pgfsetmiterjoin
\pgfsetbuttcap
\definecolor{dialinecolor}{rgb}{0.000000, 0.000000, 0.000000}
\pgfsetstrokecolor{dialinecolor}
\draw (7.964000\du,11.687536\du)--(7.964000\du,8.809540\du)--(13.015700\du,8.809540\du)--(13.015700\du,5.931544\du);
\pgfsetlinewidth{0.100000\du}
\pgfsetdash{}{0pt}
\pgfsetmiterjoin
\pgfsetbuttcap
\definecolor{dialinecolor}{rgb}{0.000000, 0.000000, 0.000000}
\pgfsetstrokecolor{dialinecolor}
\draw (17.238400\du,10.521236\du)--(17.238400\du,8.226390\du)--(13.015700\du,8.226390\du)--(13.015700\du,5.931544\du);
\definecolor{dialinecolor}{rgb}{1.000000, 1.000000, 1.000000}
\pgfsetfillcolor{dialinecolor}
\fill (-14.887300\du,5.169620\du)--(-14.887300\du,6.969620\du)--(-11.562300\du,6.969620\du)--(-11.562300\du,5.169620\du)--cycle;
\pgfsetlinewidth{0.100000\du}
\pgfsetdash{}{0pt}
\pgfsetmiterjoin
\definecolor{dialinecolor}{rgb}{0.000000, 0.000000, 0.000000}
\pgfsetstrokecolor{dialinecolor}
\draw (-14.887300\du,5.169620\du)--(-14.887300\du,6.969620\du)--(-11.562300\du,6.969620\du)--(-11.562300\du,5.169620\du)--cycle;
% setfont left to latex
\definecolor{dialinecolor}{rgb}{0.000000, 0.000000, 0.000000}
\pgfsetstrokecolor{dialinecolor}
\node at (-13.224800\du,6.269620\du){Musée};
\definecolor{dialinecolor}{rgb}{1.000000, 1.000000, 1.000000}
\pgfsetfillcolor{dialinecolor}
\fill (-4.827890\du,4.955100\du)--(-2.480390\du,3.546600\du)--(-0.132890\du,4.955100\du)--(-2.480390\du,6.363600\du)--cycle;
\pgfsetlinewidth{0.100000\du}
\pgfsetdash{}{0pt}
\pgfsetmiterjoin
\definecolor{dialinecolor}{rgb}{0.000000, 0.000000, 0.000000}
\pgfsetstrokecolor{dialinecolor}
\draw (-4.827890\du,4.955100\du)--(-2.480390\du,3.546600\du)--(-0.132890\du,4.955100\du)--(-2.480390\du,6.363600\du)--cycle;
% setfont left to latex
\definecolor{dialinecolor}{rgb}{0.000000, 0.000000, 0.000000}
\pgfsetstrokecolor{dialinecolor}
\node[anchor=east] at (-5.127890\du,4.655100\du){0:n};
\definecolor{dialinecolor}{rgb}{0.000000, 0.000000, 0.000000}
\pgfsetstrokecolor{dialinecolor}
\node[anchor=west] at (0.167110\du,4.655100\du){1:1};
\definecolor{dialinecolor}{rgb}{0.000000, 0.000000, 0.000000}
\pgfsetstrokecolor{dialinecolor}
\node at (-2.480390\du,5.155100\du){possède};
\pgfsetlinewidth{0.100000\du}
\pgfsetdash{}{0pt}
\pgfsetmiterjoin
\pgfsetbuttcap
\definecolor{dialinecolor}{rgb}{0.000000, 0.000000, 0.000000}
\pgfsetstrokecolor{dialinecolor}
\draw (-0.082924\du,4.955100\du)--(5.513656\du,4.955100\du)--(5.513656\du,4.981080\du)--(11.110235\du,4.981080\du);
\pgfsetlinewidth{0.100000\du}
\pgfsetdash{}{0pt}
\pgfsetmiterjoin
\pgfsetbuttcap
\definecolor{dialinecolor}{rgb}{0.000000, 0.000000, 0.000000}
\pgfsetstrokecolor{dialinecolor}
\draw (-11.511882\du,6.069620\du)--(-8.194869\du,6.069620\du)--(-8.194869\du,4.955100\du)--(-4.877856\du,4.955100\du);
\definecolor{dialinecolor}{rgb}{1.000000, 1.000000, 1.000000}
\pgfsetfillcolor{dialinecolor}
\pgfpathellipse{\pgfpoint{-7.130390\du}{-3.703400\du}}{\pgfpoint{2.347500\du}{0\du}}{\pgfpoint{0\du}{0.900000\du}}
\pgfusepath{fill}
\pgfsetlinewidth{0.100000\du}
\pgfsetdash{}{0pt}
\definecolor{dialinecolor}{rgb}{0.000000, 0.000000, 0.000000}
\pgfsetstrokecolor{dialinecolor}
\pgfpathellipse{\pgfpoint{-7.130390\du}{-3.703400\du}}{\pgfpoint{2.347500\du}{0\du}}{\pgfpoint{0\du}{0.900000\du}}
\pgfusepath{stroke}
% setfont left to latex
\definecolor{dialinecolor}{rgb}{0.000000, 0.000000, 0.000000}
\pgfsetstrokecolor{dialinecolor}
\node at (-7.130390\du,-3.503400\du){adresse};
\pgfsetlinewidth{0.100000\du}
\pgfsetdash{}{0pt}
\pgfsetmiterjoin
\pgfsetbuttcap
\definecolor{dialinecolor}{rgb}{0.000000, 0.000000, 0.000000}
\pgfsetstrokecolor{dialinecolor}
\draw (-11.511882\du,6.069620\du)--(-10.520032\du,6.069620\du)--(-10.520032\du,-3.703400\du)--(-9.528183\du,-3.703400\du);
\definecolor{dialinecolor}{rgb}{1.000000, 1.000000, 1.000000}
\pgfsetfillcolor{dialinecolor}
\pgfpathellipse{\pgfpoint{-13.750400\du}{-4.303400\du}}{\pgfpoint{1.577500\du}{0\du}}{\pgfpoint{0\du}{0.900000\du}}
\pgfusepath{fill}
\pgfsetlinewidth{0.100000\du}
\pgfsetdash{}{0pt}
\definecolor{dialinecolor}{rgb}{0.000000, 0.000000, 0.000000}
\pgfsetstrokecolor{dialinecolor}
\pgfpathellipse{\pgfpoint{-13.750400\du}{-4.303400\du}}{\pgfpoint{1.577500\du}{0\du}}{\pgfpoint{0\du}{0.900000\du}}
\pgfusepath{stroke}
% setfont left to latex
\definecolor{dialinecolor}{rgb}{0.000000, 0.000000, 0.000000}
\pgfsetstrokecolor{dialinecolor}
\node at (-13.750400\du,-4.103400\du){nom};
\definecolor{dialinecolor}{rgb}{1.000000, 1.000000, 1.000000}
\pgfsetfillcolor{dialinecolor}
\pgfpathellipse{\pgfpoint{-1.315400\du}{-0.803400\du}}{\pgfpoint{1.962500\du}{0\du}}{\pgfpoint{0\du}{0.900000\du}}
\pgfusepath{fill}
\pgfsetlinewidth{0.100000\du}
\pgfsetdash{}{0pt}
\definecolor{dialinecolor}{rgb}{0.000000, 0.000000, 0.000000}
\pgfsetstrokecolor{dialinecolor}
\pgfpathellipse{\pgfpoint{-1.315400\du}{-0.803400\du}}{\pgfpoint{1.962500\du}{0\du}}{\pgfpoint{0\du}{0.900000\du}}
\pgfusepath{stroke}
% setfont left to latex
\definecolor{dialinecolor}{rgb}{0.000000, 0.000000, 0.000000}
\pgfsetstrokecolor{dialinecolor}
\node at (-1.315400\du,-0.603400\du){ville};
\pgfsetlinewidth{0.100000\du}
\pgfsetdash{}{0pt}
\pgfsetmiterjoin
\pgfsetbuttcap
\definecolor{dialinecolor}{rgb}{0.000000, 0.000000, 0.000000}
\pgfsetstrokecolor{dialinecolor}
\draw (-13.224800\du,5.119156\du)--(-13.224800\du,2.633110\du)--(-1.315400\du,2.633110\du)--(-1.315400\du,0.147064\du);
\pgfsetlinewidth{0.100000\du}
\pgfsetdash{}{0pt}
\pgfsetmiterjoin
\pgfsetbuttcap
\definecolor{dialinecolor}{rgb}{0.000000, 0.000000, 0.000000}
\pgfsetstrokecolor{dialinecolor}
\draw (-13.224800\du,5.119156\du)--(-13.224800\du,0.883110\du)--(-13.750400\du,0.883110\du)--(-13.750400\du,-3.352936\du);
\definecolor{dialinecolor}{rgb}{1.000000, 1.000000, 1.000000}
\pgfsetfillcolor{dialinecolor}
\pgfpathellipse{\pgfpoint{48.438400\du}{-3.386400\du}}{\pgfpoint{1.577500\du}{0\du}}{\pgfpoint{0\du}{0.900000\du}}
\pgfusepath{fill}
\pgfsetlinewidth{0.100000\du}
\pgfsetdash{}{0pt}
\definecolor{dialinecolor}{rgb}{0.000000, 0.000000, 0.000000}
\pgfsetstrokecolor{dialinecolor}
\pgfpathellipse{\pgfpoint{48.438400\du}{-3.386400\du}}{\pgfpoint{1.577500\du}{0\du}}{\pgfpoint{0\du}{0.900000\du}}
\pgfusepath{stroke}
% setfont left to latex
\definecolor{dialinecolor}{rgb}{0.000000, 0.000000, 0.000000}
\pgfsetstrokecolor{dialinecolor}
\node at (48.438400\du,-3.186400\du){nom};
\definecolor{dialinecolor}{rgb}{1.000000, 1.000000, 1.000000}
\pgfsetfillcolor{dialinecolor}
\pgfpathellipse{\pgfpoint{42.727200\du}{-7.191860\du}}{\pgfpoint{2.347500\du}{0\du}}{\pgfpoint{0\du}{0.900000\du}}
\pgfusepath{fill}
\pgfsetlinewidth{0.100000\du}
\pgfsetdash{}{0pt}
\definecolor{dialinecolor}{rgb}{0.000000, 0.000000, 0.000000}
\pgfsetstrokecolor{dialinecolor}
\pgfpathellipse{\pgfpoint{42.727200\du}{-7.191860\du}}{\pgfpoint{2.347500\du}{0\du}}{\pgfpoint{0\du}{0.900000\du}}
\pgfusepath{stroke}
% setfont left to latex
\definecolor{dialinecolor}{rgb}{0.000000, 0.000000, 0.000000}
\pgfsetstrokecolor{dialinecolor}
\node at (42.727200\du,-6.991860\du){période};
\definecolor{dialinecolor}{rgb}{1.000000, 1.000000, 1.000000}
\pgfsetfillcolor{dialinecolor}
\pgfpathellipse{\pgfpoint{30.486500\du}{-7.191860\du}}{\pgfpoint{4.080000\du}{0\du}}{\pgfpoint{0\du}{0.900000\du}}
\pgfusepath{fill}
\pgfsetlinewidth{0.100000\du}
\pgfsetdash{}{0pt}
\definecolor{dialinecolor}{rgb}{0.000000, 0.000000, 0.000000}
\pgfsetstrokecolor{dialinecolor}
\pgfpathellipse{\pgfpoint{30.486500\du}{-7.191860\du}}{\pgfpoint{4.080000\du}{0\du}}{\pgfpoint{0\du}{0.900000\du}}
\pgfusepath{stroke}
% setfont left to latex
\definecolor{dialinecolor}{rgb}{0.000000, 0.000000, 0.000000}
\pgfsetstrokecolor{dialinecolor}
\node at (30.486500\du,-6.991860\du){texte descriptif};
\pgfsetlinewidth{0.100000\du}
\pgfsetdash{}{0pt}
\pgfsetmiterjoin
\pgfsetbuttcap
\definecolor{dialinecolor}{rgb}{0.000000, 0.000000, 0.000000}
\pgfsetstrokecolor{dialinecolor}
\draw (46.810503\du,-3.386400\du)--(44.989180\du,-3.386400\du)--(44.989180\du,-1.513520\du)--(43.167857\du,-1.513520\du);
\pgfsetlinewidth{0.100000\du}
\pgfsetdash{}{0pt}
\pgfsetmiterjoin
\pgfsetbuttcap
\definecolor{dialinecolor}{rgb}{0.000000, 0.000000, 0.000000}
\pgfsetstrokecolor{dialinecolor}
\draw (30.486500\du,-6.241396\du)--(30.486500\du,-4.352690\du)--(38.952600\du,-4.352690\du)--(38.952600\du,-2.463984\du);
\pgfsetlinewidth{0.100000\du}
\pgfsetdash{}{0pt}
\pgfsetmiterjoin
\pgfsetbuttcap
\definecolor{dialinecolor}{rgb}{0.000000, 0.000000, 0.000000}
\pgfsetstrokecolor{dialinecolor}
\draw (38.952600\du,-2.463984\du)--(38.952600\du,-4.352690\du)--(42.727200\du,-4.352690\du)--(42.727200\du,-6.241396\du);
\end{tikzpicture}

  \end{adjustbox}
  \caption{Diagramme Modèle-entité-relation pour l'inventaire des oeuvres
  d'arts.}
  \end{sidewaysfigure}

  \subsection{}
  L'année d'une oeuvre doit être dans l'intervalle de naissance et de décès de
  chaque les artistes qui y a participé.

  \begin{lstlisting}[language=OCL]
  context Oeuvre
      inv: self.artistes->forAll(a1 : Artiste | 
           a.date_de_naissance.year <= self.annee <= a.date_de_deces.year)
  \end{lstlisting}

  \subsection{}
  Deux musées différents ne peuvent pas posséder le même exemplaire d'une oeuvre
  d'art.

  \begin{lstlisting}[language=OCL]
  context Oeuvre
      inv: self.exemplaires->forAll(e1 : Exemplaire, e2: Exemplaire | 
           e1.numero_exemplaire = e2.numero_exemplaire 
           implies e1.musee = e2.musee)
  \end{lstlisting}

\end{document}
